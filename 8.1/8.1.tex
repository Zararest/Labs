\documentclass[a4paper,12pt]{article} % тип документа

%  Русский язык
\usepackage[T2A]{fontenc}			% кодировка
\usepackage[utf8]{inputenc}			% кодировка исходного текста
\usepackage[english,russian]{babel}	% локализация и переносы

\usepackage{graphicx, scalerel}               % импорт изображений
\usepackage{wrapfig}                % обтекаемые изображения
\graphicspath{{pictures/}}          % обращение к подкаталогу с изображениями
\usepackage[14pt]{extsizes}         % для того чтобы задать нестандартный 14-ый размер шрифта
\usepackage[warn]{mathtext}         % русский язык в формулах
\usepackage{indentfirst}            % indent first
\usepackage[margin = 25mm]{geometry}% отступы полей
\usepackage[table,xcdraw]{xcolor}   % таблицы
\usepackage{amsmath,amsfonts,amssymb,amsthm,mathtools} % Математика
\usepackage{wasysym}                % ???
\usepackage{upgreek}                % ???  
\usepackage{caption}
\usepackage{multirow}
\captionsetup{labelsep=period}
\usepackage[font=small,labelfont=bf]{caption}
\usepackage{gensymb} % degree symbol
\usepackage{tikz}
\usetikzlibrary{positioning}


\begin{document}
	
	
	\begin{center}
		
		
		\textbf{НАЦИОНАЛЬНЫЙ ИССЛЕДОВАТЕЛЬСКИЙ УНИВЕРСИТЕТ \\ <<МОСКОВСКИЙ ФИЗИКО-ТЕХНИЧЕСКИЙ ИНСТИТУТ>>}
		\vspace{13ex}
		
		\textbf{Лабораторная работа 4.7.3\\ <<Изучение поляризованного света>>}
		\vspace{40ex}
		
		\normalsize{Шумаков Иван Игоревич \\ студент группы Б01-009\\ 3 курс ФРКТ\\}
	\end{center}
	
	\vfill 
	
	\begin{center}
		г. Долгопрудный\\ 
		2022 г.
	\end{center}
	
	
	\thispagestyle{empty} % выключаем отображение номера для этой страницы
	\newpage
	
	
	\textbf{Цель работы:} исследовать модель АЧТ; измерить постоянные Стефана-Больцама и Планка по излучению накаленных тел.
	
	\textbf{В работе используются:} оптический пирометр, модель АЧТ, исследуемые тела, вольтметры.

	\section{Теоретические сведения}
	Для измерения температуры разогретых тел, удалённых от наблюдателя, применяют методы оптической пирометрии, основанные на использовании зависимости испускательной способности исследуемого тела от температуры. Различают три температуры, функционально связанные с истинной термодинамической температурой и излучательной способностью тела: радиационную $T_{rad}$, цветовую $T_{col}$ и яркостную $T_{br}$. \par
	В работе измеряется яркостная температура. \textbf{Яркостная температура} - это температура абсолютно чёрного тела, при которой его спектральная испускательная способность равна спектральной испускательной способности исследуемого тела при той же длине волны.
	Измерение яркостной температуры раскалённого тела производится при помощи оптического пирометра с исчезающей нитью, основанного на визуальном сравнении яркости раскалённой нити с яркостью изображения исследуемого тела. \par
	Яркостная температура тела всегда ниже его термодинамической температуры. Это связано с тем, что любое нечёрное тело излучает меньше, чем АЧТ при той же температуре. Зависимость между яркостной и термодинамической температурами вольфрама приведена на рис. 1

	\begin{figure}[h]
		\centering
		\includegraphics[width=10cm]{fig2.PNG}
		\caption{График зависимости $T = f(T_{br})$ для вольфрам}
		\label{fig:vac}
	\end{figure}

	По результатам измерений мощности излучения вольфрамовой нити можно судить о справедливости закона Стефана-Больцмана. Если бы нить излучала как АЧТ, то баланс потребляемой и излучаемой энергии определялся бы соотношением 
	\begin{equation}
		W = \sigma S (T^4 - T_0^4),
	\end{equation}
	где $W$ - потребляемая нитью электрическая мощность, $S$ - площадь излучающей поверхности нити, $T$ - температура нити, $T_0$ - температура окружающей среды. Однако вольфрамовая нить излучает как серое тел, и излучение её ослаблено по сравнению с АЧТ в $\varepsilon_T$ раз для любой волны при данной температуре тела Т. Тогда предположив, что нить излучает как серое тело и с учётом того, что $T_0 \ll T$, выражение (1) можно переписать в виде
	\begin{equation}
		W = \varepsilon_T S \sigma T^4
	\end{equation}
	В справедливости закона Стефана-Больцмана можно убедиться, построив график зависимости $W(T)$ в логарифмическом масштабе и по углу наклона определить показатель степени $n$ исследуемой температурной зависимости. В пределах погрешности показатель степени должен быть близок к четырём. \par
	Также из формулы (2) можно определить постоянную Стефана-Больцмана.

	\section{Экспериментальная установка}

	\begin{figure}[h]
		\centering
		\includegraphics[width=11cm]{fig1.PNG}
		\caption{Схема экспериментальной установки: 1 - блок питания; 2 - тумблер включения питания образцов; 3 - тумблер нагрева нити пирометра; 4 - кнопка "Нагрев нити"; 5 - кнопка "охлаждение нити"; 6 - тумблер переключения образцов; 7 - регулятор мощности нагрева образцов; 8 - окуляр пирометра; 9 - корпус пирометра; 10 - объектив пирометра; 11 - переключение диапазонов; 12 - ручка смещения красного светофильтра; 13 - регулировочный винт; 14 - вольтметр (напряжение на лампе накаливания); 15 - амперметр (ток через образцы); 16 - вольтметр в цепи термопары; 17 - модель АЧТ; 18 трубка с кольцами из материалов с различной излучательной способностью; 19 - лампа накаливания; 20 - неоновая лампочка}
		\label{fig:vac}
	\end{figure}

	Исследуемые в работе образцы:
	\begin{itemize}
		\item \textbf{модель абсолютно чёрного тела} - керамическая трубка, закрытая с одного конца и окружённая для теплоизоляции внешним кожухом. Температура в трубке измеряется с помощью термопары хромель-алюмель
		\item \textbf{керамическая трубка с набором колец из различных материалов}, нагреваемая изнутри нихромовой спиралью. Материалы колец имеют различную излучательную способность
		\item \textbf{вольфрамовая нить электрической лампочки}
	\end{itemize}

	\newpage
	
\end{document}