\documentclass[a4paper,12pt]{article} % тип документа

%  Русский язык
\usepackage[T2A]{fontenc}			% кодировка
\usepackage[utf8]{inputenc}			% кодировка исходного текста
\usepackage[english,russian]{babel}	% локализация и переносы

\usepackage{graphicx, scalerel}               % импорт изображений
\usepackage{wrapfig}                % обтекаемые изображения
\graphicspath{{pictures/}}          % обращение к подкаталогу с изображениями
\usepackage[14pt]{extsizes}         % для того чтобы задать нестандартный 14-ый размер шрифта
\usepackage[warn]{mathtext}         % русский язык в формулах
\usepackage{indentfirst}            % indent first
\usepackage[margin = 25mm]{geometry}% отступы полей
\usepackage[table,xcdraw]{xcolor}   % таблицы
\usepackage{amsmath,amsfonts,amssymb,amsthm,mathtools} % Математика
\usepackage{wasysym}                % ???
\usepackage{upgreek}                % ???  
\usepackage{caption}
\usepackage{multirow}
\captionsetup{labelsep=period}
\usepackage[font=small,labelfont=bf]{caption}
\usepackage{gensymb} % degree symbol
\usepackage{tikz}
\usetikzlibrary{positioning}


\begin{document}
	
	
	\begin{center}
		
		\textbf{НАЦИОНАЛЬНЫЙ ИССЛЕДОВАТЕЛЬСКИЙ УНИВЕРСИТЕТ \\ <<МОСКОВСКИЙ ФИЗИКО-ТЕХНИЧЕСКИЙ ИНСТИТУТ>>}
		\vspace{13ex}
		
		\textbf{Лабораторная работа 5.2.1\\ <<Опыт Франка-Герца>>}
		\vspace{40ex}
		
		\normalsize{Шумаков Иван Игоревич \\ студент группы Б01-009\\ 3 курс ФРКТ\\}
	\end{center}
	
	\vfill 
	
	\begin{center}
		г. Долгопрудный\\ 
		2022 г.
	\end{center}
	
	
	\thispagestyle{empty} % выключаем отображение номера для этой страницы
	\newpage


    Данные по графикам
    максимум, минимум, максимум, минимум
    1) 9В - (21.6	0.236) (26.5	0.071) (38.6	0.255) (50.1	0.192)
    2) 6В - (20.5   0.263) (24.8	0.144) (39.0	0.347) (46.1	0.305)
    3) 4В - (20.5   0.274) (23.3	0.211) (37.9    0.419) (42.7	0.387)

\end{document}